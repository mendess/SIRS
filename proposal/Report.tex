\documentclass[a4paper]{article}
\usepackage[utf8x]{inputenc}
\usepackage[english]{babel}
\usepackage{a4wide}
\usepackage[pdftex,hidelinks]{hyperref}
\usepackage{float}
\usepackage{graphicx}
\usepackage{fancyhdr}
\usepackage{booktabs}
\usepackage{lastpage}
\usepackage{array}
\usepackage{tikz}

\usetikzlibrary{arrows,positioning,automata,decorations.markings,shadows,shapes,calc}

\pagestyle{fancy}
\fancyhf{}

\rfoot{Page \thepage~of~\pageref{LastPage}}
\newcommand{\centered}[1]{\begin{tabular}{l} #1 \end{tabular}}

\begin{document}

\title{Spy Kid}
\author{%
    \begin{tabular}{lll}
        \centered{95145} & \centered{Eva Verboom} &
        \centered{\includegraphics[width=3cm]{images/eva.jpeg}} \\
        \centered{95383} & \centered{Felipe Gorostiaga} &
        \centered{\includegraphics[width=3cm]{images/felipe.jpg}} \\
        \centered{97144} & \centered{Pedro Mendes} &
        \centered{\includegraphics[width=3cm]{images/pedro.jpg}} \\
    \end{tabular}
}

\date{\today}

\begin{titlepage}

    %título
    \begin{center}
        \begin{minipage}{0.75\linewidth}
            \centering
            \vspace{1.5cm}
            %títulos
            \href{https://fenix.tecnico.ulisboa.pt/disciplinas/SIRS7/2019-2020/1-semestre}
            {\scshape\LARGE Network and Computer Security} \par
            \vspace{1cm}
            {\scshape\Large Alameda} \par
            \vspace{1cm}
            {\scshape\Large Group 28} \par
            \vspace{1.5cm}

            \maketitle
        \end{minipage}
    \end{center}

\end{titlepage}

\tableofcontents
\thispagestyle{empty}
\newpage
\setcounter{page}{1}

\pagebreak

%% I  Problem (Given the chosen scenario, where is security necessary? What is the main problem
%% being solved? Use around 200 words)
\section{The Problem}

During this project we will be working on the development of a secure child locator. As a guardian,
one might want to track their children to ensure that they aren't straying too far from where they
are supposed to be and in case something is wrong, to find them. Of course, this is very sensitive
information, because you do not want malicious actors to be able to track your children and misuse
the information in any way. Therefore, accessing the information should only be possible for
authorised guardian and secure communication of the child's location is required as well as secure
storage on remote servers. We will tackle the security issues in the same order, first we will
ensure that the application only allows authorised users to access the information they are allowed
to access and then we will work on secure communication protocols. When this is ensured, we will
direct our focus on the servers where the data is stored, assuming that the server is
``honest but curious''.

%% a. Requirements (Which security requirements were identified for the solution? Present as a list)
\subsection{Requirements}

The security requirements needed to solve this problem are then:
\begin{itemize}
    \item The system must maintain the confidentially of all data that is classified as
        confidential;
    \item The system must identify users in a reliable way;
    \item The system must only show data to users if they are authorised to access that particular
        data;
    \item The system must not allow anyone to fake localisation data;
    \item The system must add time-stamps to the location broadcasted by the child's device;
    \item The system must use secure communication channels to exchange data between the server and
        the clients.
\end{itemize}

%% b. Trust assumptions (Be explicit about trust relationships. Who will be fully trusted,
%% partially trusted, or untrusted)
\subsection{Trust Assumptions}
There are three groups of actors to be distinguished here: the children, the guardians and the
external servers. Both children and guardians shall be fully trusted by the system (though only
after authentication) and the external servers shall be partially trusted. All other actors that can
in some way have access to the data are untrusted parties.

%% Proposed solution (overview with diagram and explanation with around 200 words or less)
\section{Proposed Solution}

%% a) Deployment (describe distinct machines and how they will be interconnected)
\subsection{Deployment}

The system will consist of 3 moving parts:

\begin{itemize}
    \item A centralised server (may or may not be a distributed system);
    \item Multiple guardian clients;
    \item Multiple child clients.
\end{itemize}

A graphical representation of the machines and the way they are interconnected is shown in
\autoref{fig:graph}.

\begin{figure}
    \centering
    \tikzstyle{struct}=[%
    fill=white,
    draw=black,
    rectangle,
    drop shadow,
    minimum height=1.5cm,
    minimum width=1cm
]

\begin{tikzpicture}[font=\ttfamily,auto]
    \node (Server) [struct]{\textbf{\Large{Server}}};
    \node[below left = of Server] (Child) [struct]{\textbf{\Large{Child}}};
    \node[below right = of Server] (Guardian) [struct]{\textbf{\Large{Guardian}}};

    \draw[->, very thick, to path={|- (\tikztotarget)}] (Child.north) to (Server.west)
    node[above left = 0.3cm and 0.1cm] {\texttt{location beacon}};

    \draw[->, very thick, to path={|- (\tikztotarget)}] (Guardian.north) to (Server.east)
    node[above right = 0.3cm and 0.1cm] {\texttt{child location requests}};
    \draw[->, very thick, to path={|- (\tikztotarget)}] (Server.south) to (Guardian)
    node[below right = 2cm and -2.1cm of Server] {\texttt{child locations}};

\end{tikzpicture}

    \caption{Diagram of the System}\label{fig:graph}
\end{figure}

To achieve secure communication from the child and guardian clients both will require appropriate
methods of authentication.

%% Secure channel(s) to configure (identify communication entities; existing library/tool to use;
%% what keys will exist and how will they be distributed)
\subsection{Secure channels}

The guardian will have to establish a public and private key pair with the server to ensure
confidentiality and a local only password to use the app (maybe also making use of fingerprint
scanning).

The child and guardian clients need to agree on a private key that only they know, to encrypt every
piece of data they send to the server. In order for the agreement of the keys to not be intercepted
this agreement has to be made physically, through the scanning of a QR code or other similar method.

Afterwards the child client also establishes a secure channel with the server using the same method
the guardian client uses.


%% Secure protocol(s) to develop (identify communication entities; language to use for
%% implementation; what keys will exist and how will they be distributed)
\subsection{Secure Protocols}
Effectively the entities that will interact in the system are only the child and guardian clients,
the server will only serve safe remote storage for the data. By using public private key pairs for
communication with the server we ensure authenticity and by using private key encryption for the
data we ensure confidentiality and integrity of the data, meaning the server can never read the
sensitive data sent to it.

The other task of the server is to pair guardians with their children making sure that no guardian
can ever request and get data from a child other than their own, despite the fact that they would
need to know that child's symmetric key to decrypt the data.

\section{Plan}

%% Versions (Describe basic, intermediate and advanced versions of the work and when are they
%% expected to be achieved)
\subsection{Project Versions}


\subsubsection{Minimum Viable Product (MVP)}
The MVP of our product should consist of the following:
\begin{itemize}
    \item Guardian accounts which can be authenticated by the server;
    \item Guardians should be able to create accounts for their children;
    \item Child clients should be able to localise themselves outdoors;
    \item A central server that communicates with all clients and authorises guardians to view the
        tracking data of their children;
    \item Secure channels for data communication between the server and all clients, using library
        protocols.
\end{itemize}

\subsubsection{Intermediate}
The intermediate version of our product should include:
\begin{itemize}
    \item Child clients should also be able to locate themselves indoors;
    \item Child clients should be able to send alerts, when they press an SOS button or go out of
        safe zones;
    \item Guardians should know when a child broadcasted a certain location;
    \item Guardian clients should be able to receive alerts send by the child and when the child did
        not broadcast for a certain period of time.
    \item Allow multiple guardians.
\end{itemize}

\subsubsection{Full Feature Release}
Finally, the full feature release will extend the product with:
\begin{itemize}
    \item Login as a user should be done using two-factor authentication;
    \item Data should be send and stored encrypted, such that only guardians are able to track the
        child and the server is unable to know the location;
    \item Provide history of the child's location.
\end{itemize}

%% Effort commitments (table containing one row per week until the submission date; and one
%% column per group member with expected activities for the given week; some cells may be left
%% blank because of work in other courses)
\subsection{Effort Commitments}
The table below describes the activities each member of the team commits to during the project. This
division of tasks is general and all team members also commit to helping each other where needed.
Changes to the schedule during the project are likely, though every team members holds
responsibility for his/her own subjects.
\begin{center}
\begin{tabular}{c m{3.7cm} m{3.7cm} m{3.7cm}}
    \toprule
    Week & Eva & Felipe & Pedro\\ \toprule
    28/10---3/11 & Basic non secure child client & Basic non secure guardian client & Basic non secure server \\\midrule
    4/11---10/11 & Basic secure child client and testing & Basic secure guardian client and testing & Basic secure server and testing\\\midrule
    11/11---17/11 & Intermediate child client & Intermediate guardian client & Intermediate secure server\\\midrule
    18/11---24/11 & Test intermediate child client & Test intermediate guardian client & Test intermediate secure server\\\midrule
    25/11---1/12 & Advanced child client and general outlines report & Advanced guardian client & Advanced server\\\midrule
    2/12---8/12 & Testing final product & Testing final product & Testing final product\\\midrule
    9/12---15/12 & Finishing report & Finishing report & Finishing report\\\bottomrule
\end{tabular}
\end{center}

%% References (tools, libraries, etc. that will be used in the project. State if each tool has been
%% found/installed/tested at the time of proposal)
\section{References}
The main technology for developing the applications will be the android studio and the built in Java
encryption libraries (which have been tested at the time of writing this proposal).

For the server we intend to use Postgres as the database and Rust as the programming language due to
being a memory safe and high performing language (which have also been installed and tested to work).

\end{document}

